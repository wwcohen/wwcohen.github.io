\documentstyle[11pt]{article}

\setlength{\textheight}{9.25in}
\setlength{\textwidth}{7in}
\setlength{\oddsidemargin}{-0.5in}
\setlength{\evensidemargin}{-0.5in}
\setlength{\topmargin}{-0.5in}
\renewcommand{\topfraction}{0.99}
\renewcommand{\bottomfraction}{0.99}
\renewcommand{\textfraction}{0.01}

\newcommand{\prt}[1]{\vspace{\baselineskip}

{\noindent \bf #1:}}
\newcommand{\prti}[2]{\vspace{\baselineskip}

\noindent {\bf #1.} #2}
\newcommand{\bi}{\begin{itemize}}
\newcommand{\ei}{\end{itemize}}
\newcommand{\bd}{\begin{description}}
\newcommand{\ed}{\end{description}}

\begin{document}

\begin{center}
{\bf William W. Cohen}\\
Center for Automated Learning and Discovery\\
Carnegie-Mellon University\\
Pittsburgh, PA, 15213
\end{center}

\prt{Professional Experience}
\bi
\item May 2003--present. Associate Research Professor, Center for 
Automated Learning and Discovery, Carnegie-Mellon University,
Pittsburgh, PA. 
\item July 2002--May 2003. Visiting Associate Professor, Center for 
Automated Learning and Discovery, Carnegie-Mellon University,
Pittsburgh, PA.  
\item April 2000--May, 2002. Distinguished Research Scientist,
Whizbang Labs, Pittsburgh, PA. 
\item September 1990---April 2000. Member Technical Staff, AT\&T Bell
Laboratories, Murray Hill NJ; Principle Research Staff Member, AT\&T
Labs-Research, Florham Park NJ.
\ei

\prt{Education}

\prti{Ph.D. in Computer Science}{Rutgers University, New Brunswick, New
Jersey.  August 1990. Doctoral thesis, {\it
Explanation Based Generalization as an Abstraction Mechanism for
Concept Learning}, under the direction of Dr. Alex Borgida.}

\prti{M.S. in Computer Science}{Rutgers University, New Brunswick, New
Jersey. May 1988.}

\prti{B.S. in Computer Science}{Duke University, Durham, North Carolina.
May 1984.}

\prt{Relevant publications}

\bd
\item[2002] W.~Cohen,  M.~Hurst and L.~Jensen, 
	``A Flexible Learning System for Wrapping Tables and Lists in
	HTML Documents'', in {\em Proceedings of the Eleventh
	International World Wide COnference (WWW-2002)}.

\item[1998] W.~Cohen, ``Hardness Results for Learning 
		First-Order Representations and Programming by 
		Demonstration'', {\it Machine Learning}, 30(1),
	        pp 57--88.

\item[1994] 	W.~Cohen, ``Grammatically Biased Learning:
		Learning Logic Programs Using an Explicit 
		Antecedent Description Language'', 
		{\em Artificial Intelligence}, Vol 68, pp 303-366.
\ed

\prt{Other representative publications (of 93)}

\bd
\item[1999] W.~Cohen, Y.~Singer,
	``Context-sensitive learning methods for text categorization'',
	in {\it ACM Transactions on Information Systems}, 17(2), Apr 1999,
	pages 171-173.

\item[1999] W.~Cohen and Y.~Singer, ``A Simple, Fast, and Effective
	Rule Learner'', in {\em Proceedings, 
	Seventeenth National Conference on
	Artificial Intelligence} (AAAI-99).

\item[1999] W.~Cohen, Y.~Singer, R.~Schapire,
	``Learning to Order Things'', 	in
	{\it Journal of Artificial Intelligence Research},
	10, 1999, pp 243-270. 

\item[1995] W.~Cohen, ``Fast Effective Rule Induction'', 
		in {\em  Machine Learning, Proceedings
		of the Twelfth International Conference} (ICML-95).
	(There are 177 citations to this paper in ResearchIndex.)
\ed


\prt{Professional Service}
\bi
\item November 2001---present.  Director of the
International Machine Learning Society.  
\item September 2001---present.  Action Editor for {\it The Journal of Machine Learning Research\/}.
\item January 1997---September 2001. Action Editor for the journal {\it Machine Learning\/}.
\item January 1995---December 1997. Associate Editor for the 
{\it Journal of Artificial Intelligence Research}.
\item March 1998.  With Jaime Carbonell and Yiming Yang (of CMU),
editor of special issue of the journal {\it Machine Learning\/} on the
topic ``{machine learning and information retrieval\/}''.
\item Co-chair of the 1994 International Machine Learning Conference.
\item Area chair for ML-2000, SIGIR-2001, and SIGIR-2002; Senior PC member, AAAI-2004.
\item Member of program committees for KDD-2003, SIGMOD-2002, NIPS-2002, ICML-2002,
ICML-2001, SIGIR-2001, WWW-2000, ILP-2000, SIGIR-99, WWW-99, ILP-99,
COLT-98, ML-97, ILP-97, AAAI-96, ALT-96, ILP-95, ILP-94, AAAI-93, and
ML-93, and various specialized workshops.
\ei

\prt{Invited Talks and Seminars} 
\bi
\item October 2002, invited talk, ``Exploiting Document Structure in
Information Extraction and Document Classification'', McKay
Distinguished Lecture, University of California, Berkeley.
\item October 2001, invited talk, ``Issues in Extracting Information
from the Web'', at the 7th International Workshop on Parsing
Technologies, Sponsored by ACL/SIGPARSE, Bejing, China.
\item December 2000, invited talk, ``Learning Using the Web as Background Knowledge'',
at the Eleventh International Conference on Algorithmic Learning
Theory, Sydney, Australia.
\item June 1999, keynote address, ``What can we learn from the Web?''
at the 16th International Conference on Machine Learning, Bled,
Slovenia.
\item March 1996, invited talk, ``What the Well-Informed IR
Researcher Should Know About Machine Learning'', at the 1996
AAAI Spring Symposium on Machine Learning and Information Access,
Palo Alto, CA.
\item September 1995, invited talk, 
``Learning to Classify English Text with ILP
Methods'', at the Fifth International Workshop on Inductive Logic
Programming, Leuven, Belgium.
\ei

\prt{Patents and Patent Applications}
\bi
\item {\em A system and method for accessing heterogeneous databases}.
Patent \#6,295,533.
\item {\em A system and method for finding information in a
distributed information system using query learning
and meta search}.  Patent \# 5,347,623.
\item {\em Rule induction on large noisy data sets}.  Patent \# 5,719,692.
\item {\em Software discovery system}.  Patent \#  5,642,472. 
\item {\em Biased learning system}.  Patents \# 5,481,650 and \# 5,627,945.
\ei


\end{document}
